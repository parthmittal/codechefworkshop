\documentclass{article}

\usepackage{listings}
\usepackage{mathtools}
\usepackage{amssymb}
\usepackage{amsmath}
\usepackage{amsthm}
\usepackage{xcolor}
\usepackage[colorlinks = true,
            linkcolor = blue,
            urlcolor  = blue,
            citecolor = blue,
            anchorcolor = blue]{hyperref}
\newcommand{\MYhref}[3][blue]{\href{#2}{\color{#1}{#3}}}%

\usepackage{amsfonts}

\DeclarePairedDelimiter{\braces}{(}{)}

\title{Sieves}
\setlength{\parindent}{0cm}
\definecolor{darkgreen}{rgb}{0,0.6,0}

\newtheorem{theorem}{Theorem}
\newtheorem{definition}{Definition}

\begin{document}
\maketitle

\lstset{language=C++,
	numbers=left,
	basicstyle=\ttfamily,
	keywordstyle=\color{blue}\ttfamily,
	stringstyle=\color{red}\ttfamily,
	commentstyle=\color{darkgreen}\ttfamily,
	morecomment=[l][\color{magenta}]{\#},
	tabsize=4
}

\section{The problem}
Compute all the primes $\leq N$.

\subsection{A trivial solution}
We can loop over all the numbers from $1$ to $N$, and for each
number, if it is prime, print it.

How many steps does this method take?
Let's assume we do the simple test for primes (running a loop from
$2$ to $m - 1$, and checking at each step whether the number divides $m$,
to check whether $m$ is prime).

This is actually a little tricky to analyse exactly, since our loop
to test for primality may terminate early the moment we find a divisor; 
we instead go for a \emph{loose} upper bound.

We know that the loop to test whether $m$ is a prime will take at most
$m - 1$ iterations, which is $O(m)$.

Adding the cost to test over each number from $1$ to $N$, we have
$O(1 + 2 + \dots + N)$, which simplifies to $O\braces*{\frac{N\times(N - 1)}2}$,
which is $O(N^2)$ (why?).

\subsection{A simple optimisation}
Do we really need to go all the way to $m - 1$ to know that $m$ is
a prime?

No, say that $q > \sqrt{m}$ is a divisor of $m$, 
then $\frac mq < \frac m{\sqrt{m}} = \sqrt{m}$.

So we only need to check for divisors upto $\sqrt{m}$.

This speeds up our solution to
$O\braces*{\sqrt{1} + \sqrt{2} + \dots + \sqrt{N}}$, which is a
little hard to analyse, so let's replace each of those terms by
$\sqrt{N}$ (so that we still have an upper bound, however loose),
and we have $O\braces*{N \times \sqrt{N}}$, which is good
improvement over $O(N^2)$.

\section{Need more speed}
Can we do better though?

So far, we've optimised the idea
\begin{lstlisting}
for (int i = 1; i <= N; ++i) {
	if (is_prime(i)) {
		cout << i << '\n';
		/* remember, '\n' is an escape sequence
		 * that prints a newline.
		 */
	}
}
\end{lstlisting}

The other way to approach this problem, is called a \emph{sieve};
instead of looking for factors of each number, we \emph{cross off}
multiples of each number.

The idea is to start off assuming that each natural number except $1$ is
prime, because we haven't found any non-trivial (different from $1$ and
the number itself) divisors of it.

Then, for each number $m > 1$, we know that $2 \times m, 3\times m, \dots$
are all definitely not prime. So we can safely \emph{cross off} these numbers.
Let's translate this idea into code.

\lstinputlisting{naive_sieve.cpp}

See an animation of a sieve (slightly more complicated than the one
above, but the general idea is similar) 
\href{https://en.wikipedia.org/wiki/File:Sieve_of_Eratosthenes_animation.gif}{here}.
Before proceeding, convince yourself that this method is correct.

Next, we analyse how many steps this takes to run.

The loop starting in line 10, clearly runs in $O(N)$.

The outer loop in line 17 runs a total of $N - 1$ times,
but the inner loop runs $\frac N1$ times on its first run,
$\frac N2$ times on its second run, $\frac N3$ times on 
the third, and so on.
So adding the overall work, we do
\[\frac N1 + \frac N2 + \dots + \frac NN\]
steps here.
Simplifying, we have
\[N \times \braces*{\frac 11 + \frac 12 + \dots + \frac 1N} =
N \times \sum_{m = 1}^N \frac 1m \]

$\sum_{m = 1}^N \frac 1m$ is called the $N^{\texttt{th}}$ harmonic
number, and is actually $O(\log N)$. (we will not prove this result here,
but a reader familiar with limits or some calculus may read this
\href{https://math.stackexchange.com/questions/306371/simple-proof-of-showing-the-harmonic-number-h-n-theta-log-n}
{thread})

This means our formula for the number of steps simplifies to $O\braces*{N \times\log N}$.

This is a significant improvement over even the optimised algorithm from earlier, since 
$\log N$ grows much more slowly than $\sqrt{N}$;
in fact, for $N = 10^6$, $N \times\log N$ is only, $6 \times 10^6$, whereas
$N \sqrt{N}$ is $10^9$.

\subsection{Exercises for even more speed}
(don't try to copy and paste the code from this file, it will not work)
\begin{itemize}
	\item
		Do we really need to check the multiples of all $m$ in the inner-loop, starting
		on line 21? (hint: use the fundamental theorem of arithmetic to argue that
		every number has at least 1 prime factor, and use this to speed up the algorithm).
	\item 
		When we are checking multiples of $m$, do we really need to start from $2 \times m$?
		Try starting with higher values, like $3 \times m$, $4 \times m$, and so on.
		Finally, try starting with $m \times m$.
\end{itemize}

Implementing both of these exercises together, you have what is known as the Sieve of Eratosthenes,
which actually runs in $O(N \log \log N)$. (the proof is again, beyond the scope of this write-up,
but the interested reader may read up on the \emph{prime harmonic series})

\end{document}
