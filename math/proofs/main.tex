\documentclass{article}

\usepackage{xcolor}
\usepackage[colorlinks = true,
            linkcolor = blue,
            urlcolor  = blue,
            citecolor = blue,
            anchorcolor = blue]{hyperref}
\newcommand{\MYhref}[3][blue]{\href{#2}{\color{#1}{#3}}}%

\usepackage{amsfonts}

\title{Proofs}
\setlength{\parindent}{0cm}

\newtheorem{theorem}{Theorem}
\newtheorem{definition}{Definition}

\begin{document}
\maketitle

\section{Introduction: What is a proof?}
Let's say one day you woke up, and somebody said to you
``The sky is yellow''.

You laugh at them.
Everybody knows that the sky is blue.

But they keep on insisting that what they say is true.

What do you do?

You go outside and check for yourself, 
(and find that the sky is still blue).

\rule{\textwidth}{0.25pt}

That is the idea of a proof.

Before we talk more formally about proofs, we need some other
words we will be using frequently, so let's learn those first.

\section{Preliminaries}

\begin{definition}[Proposition]
	A proposition is a statement that is either true or false.
\end{definition}

Well that seems pretty simple, right?

``The Earth is the third planet from the Sun''
is a proposition (which is true).

``There are $5$ prime numbers smaller than $5$''
is a proposition (which is false).

\vspace{1em}

\end{document}
